\documentclass[10pt]{amsart}
\usepackage[margin=1in]{geometry}
\usepackage{amsmath,amssymb,amsthm}

\usepackage{graphicx}
\usepackage{pstricks}
\usepackage{pst-plot}
\usepackage{wrapfig}
\usepackage{multicol}
\usepackage{hyperref}

\usepackage{fancyhdr}
\pagestyle{fancyplain}
\usepackage{lastpage}

\newcommand{\mytitle}{Similar Matrices Problems}
\newcommand{\myclass}{Math 341}

\rhead{pg. \thepage  \ of \pageref{LastPage}}    
\chead{\myclass}
\lhead{\mytitle}
\lfoot{\noindent(Draft \today)}
\cfoot{}


%The purpose of this code is to allow me to put lines in matrices so that I can create augmented matrices.
\makeatletter
\renewcommand*\env@matrix[1][*\c@MaxMatrixCols c]{%
  \hskip -\arraycolsep
  \let\@ifnextchar\new@ifnextchar
  \array{#1}}
\makeatother



\newcommand{\ds}{\displaystyle}
\begin{document}

\noindent{\huge{\bf \mytitle}}


%The following problems require the use of the Maple introduction.

\begin{enumerate}

\item[(I)] For each of the following linear transformations, find a basis $S$ for the domain and a basis $S^\prime$ for the range so that the matrix representation relative to $S$ and $S^\prime$ consists of all zeros and 1's, where the 1's occur on the diagonal of a sub matrix located in the upper left corner of the matrix.  In other words, the matrix representation consists of adding rows and/or columns of zeros to the identity matrix (where the number of 1's equals the rank of the matrix). [Hint: extend a basis for the kernel to a basis for the domain, and extend the images of these basis vectors to a basis for the range. See Examples 9 and 10 in the handout.]

\item $T(x,y,z)=(x+2y-z,y+z)$

\item $T(x,y,z)=(x+2y-z,y+z,x+3y)$

\item $T(x,y,z)=(x+2y-z,2x+4y-2z)$

\item $T(x,y)=(x+2y,y,3x-y)$

\item $T(x,y)=(x+2y,2x+4y,3x+6y)$

\item[(II)] Prove the following (which requires checking 3 things). Again, the proofs are on the first 2 pages of the handout:

\item The kernel of a linear transformation is a subspace of the domain.

\item The image of a linear transformation is a subspace of the range.

\item The eigenspace corresponding to an eigenvalue of a linear transformation is a subspace of both the domain and range.

\end{enumerate}






\end{document}







